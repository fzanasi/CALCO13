\usepackage{url}
\usepackage{vaucanson-g}
\sloppy

%\usepackage{times} %changes font to times, which saves lots of space!
\usepackage{longtable}
\usepackage{wrapfig}
\usepackage{comment}
\usepackage[small]{caption}
\setlength{\captionmargin}{12pt}

\usepackage{color}

\usepackage{graphicx,epsfig,color}

\usepackage{xypic}
\input xy
\xyoption{all}
\usepackage{verbatim}
\usepackage{amsmath}
\usepackage{amssymb}
\usepackage{eucal}
%
% --- Modifications of mathcode (cf. TeX book p. 344) ---
%
\mathcode`:="003A  % : ordinary symbol
\mathcode`;="003B  % ; ordinary symbol
\mathcode`?="003F  % ? ordinary symbol
\mathcode`|="026A  % | ordinary symbol
\mathcode`<="4268  % < abbreviates \langle
\mathcode`>="5269  % > abbreviates \rangle

\mathchardef\ls="213C    % less symbol (< used as \langle)
\mathchardef\gr="213E    % greater symbol (> used as \rangle)
\mathchardef\uparrow="0222  % adaptation mathmode of uparrow
\mathchardef\downarrow="0223  % adaptation mathmode of downarrow

\newcommand{\cbox}[1]{\vspace{0.2cm}\noindent
  \fbox{\parbox{.97\textwidth}{#1}}\vspace{0.2cm}}


\newcommand{\fG}{\mathcal{G}}    % Generic Functor G
\newcommand{\fW}{\mathcal{W}}    % Functor W
\newcommand{\fL}{\mathcal{L}}    % Functor L
\newcommand{\emp}{\epsilon}           % Empty Function


\newcommand{\lbb}{\mathopen{[\![}}
\newcommand{\rbb}{\mathclose{]\!]}}
\newcommand{\bb}[1]{\lbb #1 \rbb}

\def\pow#1{{\mathcal P_\omega}#1}
\newcommand\pto\rightharpoonup
\newcommand\E\varepsilon
\newcommand{\pp}[1]{\|#1\|}

\def\mean#1{[\![ \, #1 \, ]\!]}
\def\expr#1{<\!< \, #1 \, >\!>}
\def\ceil#1{\lceil\, #1 \,\rceil}

\def\prob#1#2{#2 \cdot #1}

\renewcommand{\cbox}[1]{\vspace{0.2cm}\noindent
  \fbox{\parbox{.97\textwidth}{#1}}\vspace{0.2cm}}

\newcommand{\pippo}[1]{\marginpar{{\bf Fil:} #1 }}    % Commento a lato, Filippo
\newcommand{\fabio}[1]{\marginpar{{\bf Fab:} #1 }}    % Commento a lato, Fabio

\renewcommand{\fG}{\mathcal{G}}    % Generic Functor G
\renewcommand{\fW}{\mathcal{W}}    % Functor W
\renewcommand{\fL}{\mathcal{L}}    % Functor L
\renewcommand{\emp}{\epsilon}           % Empty Function

%PRODUCTS
\newcommand{\setproduct}{\times} %Cartesian Product of sets and functions
\newcommand{\vectproduct}{\times} %(Bi)Product of Vector Spaces and linear maps
\newcommand{\matrixproduct}{\times} %Product of Matrices
\newcommand{\streamproduct}{\times} %Product of Streams
\newcommand{\beh}[3]{\left[\!\left[ #1 \right]\!\right]^{#2}_{#3}} % Final Morphism
\newcommand{\Beh}[3]{\Big[\!\!\Big[ #1 \Big]\!\!\Big ]^{#2}_{#3}} % Final Morphism
\newcommand{\image}{\mathrm{im}} % image
%%%% MICHELE'S MACRO
\newcommand{\comp}{\circ}               % composition of functions and linear maps
\newcommand{\dimn}{\mathrm{dim}}   % dimension
\newcommand{\transp}{{}^{\mathrm{t}}}  % transpose
\newcommand{\SR}{\mathbb{S}}            % generic field K
\newcommand{\dual}[1]{ {#1}^\star}     % dual space
\newcommand{\ddual}[1]{{#1}^{\star\star}}  % double dual space
\newcommand{\ann}{o}                    % annihilator
\newcommand{\kernel}{\mathrm{ker}} % kernel
\newcommand{\wa}{{\sc wa}}             % "wa"
\newcommand{\lwa}{{\sc lwa}}           % "lwa"
\newcommand{\Span}{\mathrm{span}}  % span
\newcommand{\R}{R}%{\mathcal{R}}    % relation
\newcommand{\mik}[1]{\marginpar{ \textbf{MiB:} {\footnotesize #1}}} % for margin notes
\newcommand{\mar}[1]{\marginpar{ \textbf{MaB:} {\footnotesize #1}}} % for margin notes



%%%%%%%%%%%%%%%%% FABIO'S MACRO

% ENVIRONMENTS

\newenvironment{todo}{\begin{itemize}\renewcommand{\labelitemi}{$\blacktriangleright$}\item \footnotesize \scshape}{\end{itemize}}
\newenvironment{Iff-RL}{\textbf{($\Rightarrow$)} }{\bigskip}
\newenvironment{Iff-LR}{\textbf{($\Leftarrow$)} }{}

\newtheorem{construction}{Construction}

% FONTS

\newcommand{\m}{\mathit}
\def \: {\colon}
\newcommand{\mb}{\mathbb}
\newcommand{\mc}{\mathcal}
\newcommand{\mf}{\mathbf}
\def \p {\mathcal P}

% FUNCTORS

\newcommand\F{\mathcal{F}}
\newcommand\V{\mathcal{V}}
\newcommand\G{\mathcal{G}}
\newcommand\D{\mathcal{D}}
\newcommand\FR{\mathcal{R}}
\newcommand\FS{\mathcal{S}}

\def \incl {\iota} % inclusion functor

% CATEGORIES

\def \catC {\mf{C}}
\def \Set {\mf{Set}}
\def \catJ {\mf{J}}
\def \catI {\mf{I}}

\newcommand{\coalg}[1]{{\bf CoAlg}(#1)}

%Lawvere theory
\def \Lw {\mathcal{L}^{\m{op}}_{\Sigma}}

%Right Kan Extension
\newcommand\U{\mathcal{U}}
\newcommand\K{\mathcal{K}}
\newcommand{\set}{\mf{Set}}
\newcommand{\prsh}[1]{\set^{#1}}

%Logic programming
\def \PP  {\widehat{\p _c}\widehat{\p _f}}
\def \At {\m{At}}
\def \SLD {\m{SLD}}
\def \Var {\m{Var}} % variables of the language
\newcommand{\Ter}[1]{\m{Ter}(#1)} % set of terms

%trees and LTSs
\def \Tand {T_{\wedge}}
\def \Tor {T_{\vee}}
\def\tr#1{\stackrel{#1}{\to}}          % Labeled Transitions 